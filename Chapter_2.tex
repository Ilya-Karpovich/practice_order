\sectionbreak \section*{
	\cyrillicfont 
	\fontsize{14pt}{0pt}\selectfont
	\englishfont 
	\redline
	2. АНАЛИЗ СУЩЕСТВУЮЩИХ РЕШЕНИЙ
}

\titlespace
\subsection*{ 
	\gostTitleFont
	\redline
	2.1 Анализ существующих решений. описание достоинств и недостатков
} 
{\gostFont
	
	\par \redline В этой главе будут проанализированы методы реализации данного веб приложения, также будут рассмотрены их плюсы и минусы. 
	
	\par \redline	1. Системы управления университетским контентом (CMS‑решения). Многие университеты используют популярные системы управления контентом (CMS), такие как \\ WordPress, Joomla или Drupal. На их основе создаются порталы, в которых публикуются новости, объявления, информация о событиях и другие материалы.
	
	\par \redline Особенности и преимущества:
	
	\par \redline • Гибкость оформления и верстки: Благодаря наличию множества тем и плагинов можно легко адаптировать внешний вид и функциональность под нужды конкретного вуза;
	
	\par \redline • Расширяемость: С помощью плагинов (например, для форумов или комментариев) можно добавить интерактивные функции, в том числе элементы чата или обсуждения;
	
	\par \redline • Простота обновления контента: Пользователи без глубоких знаний программирования могут самостоятельно редактировать и публиковать материалы.
	
	\par \redline Недостатки:
	
	\par \redline • Необходимость администрирования: Для обеспечения безопасности и стабильности системы требуется регулярное обновление плагинов и CMS, а также грамотная настройка хостинга;
	
	\par \redline • Ограниченная интеграция с реальным временем: Для реализации полноценного чата (общего или тематического) зачастую приходится прибегать к дополнительным решениям, так как из коробки CMS не всегда поддерживают обмен сообщениями в режиме реального времени. 
	
	\par \redline	2. Корпоративные и интегрированные коммуникационные платформы. Крупные образовательные учреждения могут использовать интегрированные решения, объединяющие в себе функции доски объявлений, обмена документами, видеоконференций и чата. Примерами таких систем являются Microsoft Teams, Google Classroom, Slack или специализированные платформы вроде Blackboard.
	
	\par \redline Особенности и преимущества:
	
	\par \redline • Комплексность: Платформы предоставляют широкий набор инструментов для коммуникации – от чата и видеоконференций до обмена файлами и календарной интеграции;
	
	\par \redline • Поддержка мобильных устройств: Такие решения, как правило, оптимизированы для работы на разных устройствах, что удобно для студентов и преподавателей;
	
	\par \redline • Надёжность и поддержка: Крупные коммерческие продукты предлагают регулярные обновления, техническую поддержку и высокую степень безопасности.
	
	\par \redline Недостатки:
	
	\par \redline • Ограниченная кастомизация: Возможности по адаптации интерфейса и функционала под специфические нужды конкретного университета зачастую ограничены;
	
	\par \redline • Лицензионные обязательства: Некоторые платформы требуют подписки или оплаты, что может стать препятствием для внедрения в бюджетных условиях;
	
	\par \redline • Избыточность функций: Для реализации простой доски объявлений и общего чата данные системы могут содержать избыточное количество функций, что усложняет их использование для узкоспециализированных задач. 
	
	\par \redline	3. Собственные разработки на основе веб-технологий. Еще одним направлением является разработка собственных веб-приложений, ориентированных на конкретные задачи университета. Такие решения реализуются с нуля с применением современных технологий, например, с использованием языка C\# и платформы .NET.
	
	\par \redline Особенности и преимущества:
	
	\par \redline • Полная кастомизация: Возможность реализовать точную функциональность, адаптированную под внутренние процессы вуза, без избыточного функционала;
	
	\par \redline • Интеграция с внутренними системами: Собственное решение можно интегрировать с другими информационными системами вуза (например, с базами данных, системами управления пользователями или расписанием);
	
	\par \redline • Гибкость в развитии: В процессе эксплуатации можно постепенно добавлять новый функционал, исходя из потребностей пользователей.
	
	\par \redline Недостатки:
	
	\par \redline • Затраты на разработку: Создание и поддержка собственной системы требует работы специалистов и выделения бюджетных средств на разработку, тестирование и последующую эксплуатацию;
	
	\par \redline • Вопросы безопасности: Необходимо тщательно продумывать и реализовывать меры безопасности, чтобы избежать утечек данных и обеспечить стабильную работу.   
	
	\par \redline 4. Облачные сервисы и специализированные виджеты. Существуют облачные сервисы и готовые решения, которые позволяют создать доску объявлений с минимальными усилиями. Это могут быть сервисы типа Trello, Notion или специализированные виджеты для сайтов, реализующие функции обмена новостями и комментариями.
	
	\par \redline Особенности и преимущества:
	
	\par \redline • Быстрый запуск: Такие решения, как правило, требуют минимальной настройки и позволяют быстро опубликовать информацию;
	
	\par \redline • Интуитивность использования: Готовые интерфейсы удобны даже для пользователей без технических знаний;
	
	\par \redline • Низкий порог входа: Обычно не требует значительных финансовых вложений.
	
	\par \redline Недостатки:
	
	\par \redline • Ограниченная кастомизация и интеграция: Решения часто предлагаются "из коробки" и могут не учитывать специфические особенности образовательного процесса;
	
	\par \redline • Вопросы безопасности и конфиденциальности: При использовании сторонних облачных сервисов важно обратить внимание на защиту данных и соответствие требованиям законодательства. 
	
	\par \redline Разработку системы "Доска объявлений БрГТУ" можно рассматривать как попытку объединить лучшие стороны перечисленных решений. Использование языка C\# и технологий ASP.NET позволяет получить гибкость и возможность глубокой кастомизации, а интеграция элементов новостной ленты и чата обеспечивает оперативность коммуникации среди студентов. Такой подход позволяет создать систему, адаптированную под конкретные нужды БрГТУ, что является преимуществом по сравнению с универсальными или коммерческими решениями.
	
	\par \redline Кроме того, собственная разработка позволяет решить вопросы безопасности и интеграции с существующими информационными системами вуза, что повышает уровень доверия пользователей и способствует активному использованию проекта среди студентов.
	
	\par
}
\subtitlespace

\subsection*{ 
	\gostTitleFont
	\redline
	2.2 Выбор средств реализации
} 

\subtitlespace

{\gostFont
	
	\par \redline Основные функциональные задачи, которые должен решать проект:
	
	\par \redline • Оперативное информирование студентов посредством публикации новостей и объявлений,
	
	\par \redline • Обеспечение коммуникации через чат, позволяющий в режиме реального времени обмениваться сообщениями,
	
	\par \redline • Удобство и адаптивность интерфейса для пользователей с разных устройств,
	
	\par \redline • Надежное хранение и обработка данных с возможностью масштабирования и интеграции с внутренними системами БрГТУ.
	
	\par \redline Чтобы удовлетворить этим требованиям, было решено создать собственное веб-приложение на основе современных технологий, что позволит выстроить гибкую архитектуру с возможностью дальнейшего расширения функционала. 
	
	\par \redline C\# — современный и мощный объектно-ориентированный язык программирования, который предоставляет следующие преимущества:
	
	\par \redline • Богатая стандартная библиотека и экосистема: наличие большого количества готовых компонентов и инструментов для разработки,
	
	\par \redline • Поддержка парадигмы объектно-ориентированного программирования: упрощает разработку масштабных проектов,
	
	\par \redline • Высокая читаемость и поддержка: активное сообщество разработчиков и отличная документация.
	
	\par \redline .NET Core — кроссплатформенная система разработки, отвечающая современным требованиям:
	
	\par \redline -Кроссплатформенность: возможность разворачивать приложение на Windows, \\Linux и macOS,
	
	\par \redline • Высокая производительность и масштабируемость: оптимизированный рантайм и модульная архитектура,
	
	\par \redline • Адаптивность и интеграция: легкость интеграции с различными системами и возможность использования новейших инструментов разработки.
	
	\par \redline Для реализации веб-приложения целесообразно применять некоторые из следующих технологий:
	
	\par \redline Использование архитектурного шаблона Model-View-Controller (MVC) позволяет:
	
	\par \redline • Четко разделить представление, бизнес-логику и доступ к данным,
	
	\par \redline • Облегчить сопровождение и масштабирование проекта,
	
	\par \redline • Создать чистую и легко поддерживаемую кодовую базу.
	
	\par \redline Entity Framework Core (EF Core) выступает в роли ORM-инструмента, который:
	
	\par \redline -Автоматизирует базовые операции по работе с базой данных (создание, миграция, CRUD-операции),
	
	\par \redline • Позволяет минимизировать прямое написание SQL-кода,
	
	\par \redline • Обеспечивает безопасность и целостность данных.
	
	\par \redline Для обеспечения функциональности чата в режиме реального времени планируется использовать SignalR — библиотеку для ASP.NET Core, позволяющую:
	
	\par \redline • Организовать двустороннюю связь между сервером и клиентами,
	
	\par \redline • Реализовать обмен сообщениями без необходимости постоянного обновления страницы,
	
	\par \redline • Обеспечить отзывчивость и оперативность коммуникаций в приложении.
	
	\par \redline На стороне клиента для формирования удобного и современного интерфейса будут использоваться:
	
	\par \redline • HTML5 и CSS3: для создания базовой структуры и стилизации страниц,
	
	\par \redline • JavaScript: для обеспечения интерактивности элементов приложения,
	
	\par \redline • Фреймворк Bootstrap: для быстрого создания адаптивного дизайна, корректно отображающегося на мобильных устройствах, планшетах и десктопах. 
	
	\par 
}


