\sectionbreak \section*{
	\gostTitleFont
	\redline
	ЗАКЛЮЧЕНИЕ
}

\subtitlespace

{\gostFont
	\par \redline В ходе практики была проведена комплексная исследовательская работа, направленная на анализ существующих решений в области организации информационных платформ для вузов, а также на подбор оптимальных средств реализации для будущего проекта. Было изучено множество систем – от готовых CMS и корпоративных платформ до специализированных веб-решений – что позволило оценить их преимущества и недостатки в контексте задач БрГТУ. 
	
	\par \redline Особое внимание было уделено выбору технологий для разработки собственного веб-приложения. Принятые решения включают использование языка программирования C\# и платформы .NET Core, а также таких ключевых инструментов, как ASP.NET Core MVC, Entity Framework Core и SignalR. Эти технологии обеспечивают высокую производительность, гибкость архитектурного подхода, возможность масштабирования и поддержку современных требований к безопасности и адаптивности интерфейса.
	
	\par \redline Полученные результаты практической работы подтвердили обоснованность выбранного направления, а также заложили прочный фундамент для реализации полноценного веб-приложения «Доска объявлений БрГТУ». Само приложение будет разработано в ходе дипломного проектирования, что позволит детально проработать архитектуру системы, настроить обмен информацией в чате в режиме реального времени и обеспечить эффективное взаимодействие студентов с информационными ресурсами университета.

	\par \redline Таким образом, проведенная практика способствовала не только углублению знаний и освоению современных технологий, но и определению дальнейших этапов реализации проекта, направленного на улучшение коммуникации и информационного обмена внутри БрГТУ. 
	
	\par	
}

\setcounter{subchaptercntr}{1}
\setcounter{formulacntr}{1}
\setcounter{imagecntr}{1}